%!TEX root = TableSummarizationDemo.tex

\section{Formal Description} \label{sec:formal} 

\stitle{Tables and Rules.} Our system first takes as input a relational table, which we call $\calD$. 
% For the purpose of the rest of the discussion, we will operate on this
% table $\calD$. 
We let $T$ denote the set of tuples in $\calD$, and $C$ denote 
the set of columns in $\calD$.
Our objective is to 
enable smart drill-downs on this table or on portions of it:
the result of our drill-downs are lists of {\em rules}. 
A {\em rule} is a tuple with a value for each column of the table. 
In addition, a rule has other attributes, such as count and weight associated with it. 
The value in each column of the rule can either be one of the values in the corresponding column of the table, or $\star$, representing a wild-card character representing all values in the column. A rule $r$ is said to {\em cover} a tuple $t$ from the table if all non-$\star$ values for all columns of the rule match the corresponding values in the tuple. Rule $r$ is a super-rule of $r'$ if for every non-$\star$ value in $r'$, $r$ has the same value in the same column. The {\em Count} of a rule is the number of tuples covered by that rule. 

\stitle{Rule Lists.} A {\em rule-list} is an ordered list of rules returned by our system in response to a smart drill-down operation. 
When a user drills down on a rule $r$ to know more about the part of the table covered by $r$, we display a new rule-list below $r$.
For instance, the second, third and fourth rule from Table~\ref{table:introexample} form a rule-list, which is displayed when the user clicks on the first rule. Similarly, the second, third and fourth rules in Table~\ref{table:introexample2} form a rule-list, as do the fifth, sixth and seventh rules. We now define some additional properties of rules; these properties
help us ``score'' individual rules as part of a rule-list. 

\stitle{Scoring.} There are two portions that constitute our scores for a rule as part of a rule list. 
The first portion dictates how much the rule $r$ ``covers'' the tuples in $\calD$;
the second portion dictates how ``good'' the rule $r$ is (independent of how many
tuples it covers). 
The reason why we separate the scoring into these two portions is
that they allow us to separate the inherent ``goodness'' of a rule from
how much it captures the data in $\calD$.

We now describe the first portion:
We define {\em MCount}($r, R$) (which stands for `Marginal Count') as the number of tuples covered by $r$ but not by any rule before $r$ in the rule-list $R$. A high value of $MCount$ indicates that the rule not only covers a lot of tuples, but also covers parts of the table not covered by previous rules. 

Now, onto the second portion: we let $W$ denote a function that assigns a non-negative {\em weight} to a rule based on how good the rule is, with higher weights assigned to better rules. 
The weighting function does not depend on the specific
tuples in $\calD$, but could 
depend on the number of $\star$s in $r$,
the schema of $\calD$, 
as well as the number of distinct values in each column of $\calD$. The full description of the weighting functions
can be found in~\cite{tr}.
A weighting function is said to be {\em monotonic} if for all rules $r_1$, $r_2$ such that $r_2$ is a super-rule of $r_1$, we have $W(r_1) \leq W(r_2)$; we focus
on monotonic weighting functions because we prefer 
rules that are more ``specific''
rather than those that are more ``general''. 

Thus, the total score for our list of rules is given by 
$$\text{Score}(R) = \sum_{r \in R} \underbrace{MCount(r, R)}_{\text{coverage of $r$ in $\calD$}} \times \underbrace{W(r)}_{\text{weight of $r$}}$$ 
Overall, our goal is to choose the rule-list maximizing 
total score. 
Our smart drill-downs still display the Count of each rule rather than the $MCount$. This is because while $MCount$ is useful in the rule selection process, Count is easier for a user to interpret. In any case, it would be a simple extension to display MCount in another column.

\stitle{Formal Problem:} We now formally define our problem:
\begin{problem}\label{prob:optimal-subrule-list}
Given a table $T$, a monotonic weighting function $W$, and a number $k$, find the list $R$ of $k$ rules maximizing $\text{Score}(R)$
%$$\sum_{r \in R} W(r) \times MCount(r,R)$$
such that each rule $r \in R$ is a super-rule of the user-clicked rule.
\end{problem}
