\documentclass{vldb}
\usepackage{graphicx}
\usepackage{balance}  % for  \balance command ON LAST PAGE  (only there!)
\usepackage{enumitem}
\usepackage{times}
\usepackage{subfigure}
\let\proof\relax
\let\endproof\relax
\usepackage{amsmath,amssymb,amsthm}
\usepackage{graphicx,color}
\usepackage{verbatim}
\usepackage{framed}
\usepackage[ruled,vlined]{algorithm2e}
\usepackage{framed}
\usepackage[normalem]{ulem}
\usepackage[export]{adjustbox}


\usepackage[font={small,it}]{caption}

\renewcommand{\baselinestretch}{1.0}

\renewcommand*\ttdefault{cmvtt}
\usepackage[T1]{fontenc}

% \usepackage{floatrow}
% \floatsetup[table]{font=scriptsize}
% \renewcommand\FBbskip{-10pt}


\newtheorem*{theorem*}{Theorem}
\newtheorem{theorem}{Theorem}
\newtheorem*{lemma*}{Lemma}
\newtheorem{lemma}{Lemma}
\newtheorem{definition}{Definition}
\newtheorem{example}[definition]{Example}
\newcounter{prob}
\newtheorem{problem}[prob]{Problem}

\newcommand{\agp}[1]{\textcolor{green}{Aditya: #1}}
\newcommand{\mrj}[1]{\textcolor{red}{#1}}
\newcommand{\mrjdel}[1]{\textcolor{red}{\sout{#1}}}

\newcommand{\squishlist}{
   \begin{list}{$\bullet$}
    { \setlength{\itemsep}{0pt}
      \setlength{\parsep}{2pt}
      \setlength{\topsep}{2pt}
      \setlength{\partopsep}{0pt}
    }
}
\newcommand{\stitle}[1]{\vspace{0.5em}\noindent\textbf{#1}}
\newcommand{\squishend}{\end{list}}
\newcommand{\eat}[1]{}
\newcommand{\papertext}[1]{#1}
\newcommand{\techreporttext}[1]{}


\newcommand{\calD}{\mathcal{D}\xspace}


\newenvironment{denselist}{
    \begin{list}{\small{$\bullet$}}%
    {\setlength{\itemsep}{0ex} \setlength{\topsep}{0ex}
    \setlength{\parsep}{0pt} \setlength{\itemindent}{0pt}
    \setlength{\leftmargin}{1.5em}
    \setlength{\partopsep}{0pt}}}%
    {\end{list}}

\makeatletter
\def\@copyrightspace{\relax}
\makeatother

\begin{document}

\title{Smart Drill-Down}
\numberofauthors{3} 
\author{
\alignauthor
Manas Joglekar\\
       \affaddr{Stanford University}\\
%       \affaddr{353 Serra Mall}\\
%	   \affaddr{Stanford, California 94305}\\
       \email{manasrj@stanford.edu}
\alignauthor
Hector Garcia-Molina\\
       \affaddr{Stanford University}\\
%       \affaddr{353 Serra Mall}\\
%	   \affaddr{Stanford, California 94305}\\
       \email{hector@cs.stanford.edu}
\alignauthor 
Aditya Parameswaran\\
       \affaddr{University of Illinois (UIUC)}\\
%       \affaddr{Champaign, Illinois 61820}\\
       \email{adityagp@illinois.edu}
}
\maketitle

\begin{abstract}
We present a data exploration system equipped with {\em smart drill-down},
a novel operator for interactively exploring a relational table
to discover and summarize ``interesting'' groups of tuples.
Each group of tuples is described by a {\em rule}.
For instance, the rule $(a, b, \star, 1000)$ tells us that
there are a thousand tuples with value $a$ in the first column and $b$
in the second column (and any value in the third column).
Smart drill-down presents an analyst with a list of rules that
together describe interesting aspects of the table.
The analyst can tailor the definition of interesting,
and can interactively apply smart drill-down on an existing rule to
explore that part of the table. The problems underlying smart drill-down are {\sc NP-Hard}, so our system uses an approximation algorithm for choosing the best list of rules to display. In addition, our system uses a dynamic sampling and memory management scheme to achieve low response times while interacting with large tables.
\end{abstract}

\input{intro}

%!TEX root = TableSummarizationDemo.tex

\section{Formal Description} \label{sec:formal} 

\stitle{Tables and Rules.} Our system first takes as input a relational table, which we call $\calD$. 
% For the purpose of the rest of the discussion, we will operate on this
% table $\calD$. 
We let $T$ denote the set of tuples in $\calD$, and $C$ denote 
the set of columns in $\calD$.
Our objective is to 
enable smart drill-downs on this table or on portions of it:
the result of our drill-downs are lists of {\em rules}. 
A {\em rule} is a tuple with a value for each column of the table. 
In addition, a rule has other attributes, such as count and weight associated with it. 
The value in each column of the rule can either be one of the values in the corresponding column of the table, or $\star$, representing a wild-card character representing all values in the column. A rule $r$ is said to {\em cover} a tuple $t$ from the table if all non-$\star$ values for all columns of the rule match the corresponding values in the tuple. Rule $r$ is a super-rule of $r'$ if for every non-$\star$ value in $r'$, $r$ has the same value in the same column. The {\em Count} of a rule is the number of tuples covered by that rule. 

\stitle{Rule Lists.} A {\em rule-list} is an ordered list of rules returned by our system in response to a smart drill-down operation. 
When a user drills down on a rule $r$ to know more about the part of the table covered by $r$, we display a new rule-list below $r$.
For instance, the second, third and fourth rule from Table~\ref{table:introexample} form a rule-list, which is displayed when the user clicks on the first rule. Similarly, the second, third and fourth rules in Table~\ref{table:introexample2} form a rule-list, as do the fifth, sixth and seventh rules. We now define some additional properties of rules; these properties
help us ``score'' individual rules as part of a rule-list. 

\stitle{Scoring.} There are two portions that constitute our scores for a rule as part of a rule list. 
The first portion dictates how much the rule $r$ ``covers'' the tuples in $\calD$;
the second portion dictates how ``good'' the rule $r$ is (independent of how many
tuples it covers). 
The reason why we separate the scoring into these two portions is
that they allow us to separate the inherent ``goodness'' of a rule from
how much it captures the data in $\calD$.

We now describe the first portion:
We define {\em MCount}($r, R$) (which stands for `Marginal Count') as the number of tuples covered by $r$ but not by any rule before $r$ in the rule-list $R$. A high value of $MCount$ indicates that the rule not only covers a lot of tuples, but also covers parts of the table not covered by previous rules. 

Now, onto the second portion: we let $W$ denote a function that assigns a non-negative {\em weight} to a rule based on how good the rule is, with higher weights assigned to better rules. 
The weighting function does not depend on the specific
tuples in $\calD$, but could 
depend on the number of $\star$s in $r$,
the schema of $\calD$, 
as well as the number of distinct values in each column of $\calD$. The full description of the weighting functions
can be found in~\cite{tr}.
A weighting function is said to be {\em monotonic} if for all rules $r_1$, $r_2$ such that $r_2$ is a super-rule of $r_1$, we have $W(r_1) \leq W(r_2)$; we focus
on monotonic weighting functions because we prefer 
rules that are more ``specific''
rather than those that are more ``general''. 

Thus, the total score for our list of rules is given by 
$$\text{Score}(R) = \sum_{r \in R} \underbrace{MCount(r, R)}_{\text{coverage of $r$ in $\calD$}} \times \underbrace{W(r)}_{\text{weight of $r$}}$$ 
Overall, our goal is to choose the rule-list maximizing 
total score. 
Our smart drill-downs still display the Count of each rule rather than the $MCount$. This is because while $MCount$ is useful in the rule selection process, Count is easier for a user to interpret. In any case, it would be a simple extension to display MCount in another column.

\stitle{Formal Problem:} We now formally define our problem:
\begin{problem}\label{prob:optimal-subrule-list}
Given a table $T$, a monotonic weighting function $W$, and a number $k$, find the list $R$ of $k$ rules maximizing $\text{Score}(R)$
%$$\sum_{r \in R} W(r) \times MCount(r,R)$$
such that each rule $r \in R$ is a super-rule of the user-clicked rule.
\end{problem}


%!TEX root = TableSummarizationDemo.tex

\section{Summary of Contributions}\label{sec:summary}
% say what pieces we have implemented
% algorithm for choosing rules
% samplehandler for managing memory
% our datasets


%!TEX root = TableSummarizationDemo.tex

\section{Demo Overview} \label{sec:demo} 
In our demo, we will show our prototype implementation of a system equipped with smart drill-down. The implementation includes a web interface. We have included two datasets, a marketing survey, and the US Census, which users can explore using smart drill-down. Figure~\ref{fig:interface} shows a screenshot of the web interface. 

At the top, users can set the number of new rules to display in response to every smart drill-down. The second setting is a parameter called max weight, which trades off the accuracy of the displayed rule list for processing time. Briefly, our system only considers displaying rules that have weight $\leq$ the max weight. As long as the weight of all rules in the optimal rule-list is less than this parameter, our system displays the optimal rule-list. We observe that a value of around $5$ works well for the Size weighting function. Weighting functions with higher weight values need a higher value for this parameter. 

The third parameter the can set if the weighting function. In general, our approximation algorithm works for any monotonic weighting function. But for ease of use of the web interface, we have hardcoded a few different weighting functions, that can be selected using the drop-down list in the interface. Below, the interface displays the set of columns of the database table being explored. Each column has three options: `Default', `Ignore', and `Force'. Choosing the Ignore option causes the column to be ignored, (so the weight given to a rule with a value in that column is set to that of a rule with a $\star$ value in that column). Choosing the Force option forces every displayed rule to have a non-$\star$ value in that column. 

Finally, the actual interactive table summary is displayed below. The plus and minus buttons before the rules can be used to drill-down and reverse previous drill-downs. For instance, in the figure, the user has performed a single drill down using the Size weighting function (which sets weight to the number of non-star values of a rule), and choosing the Force option for the Occupation column, and Ignore for Gender and Time in Bay Area columns. As a result, the displayed rule-list (the three rules below the first one) all have a non-$\star$ value in the Occupation column, and only $\star$ values in the Gender and Time in Bay Area columns. Unlike in a traditional drill-down, the displayed rules also have non-$\star$ values in columns other than Occupation. 

\begin{figure*}[ht]
\vspace{-5pt}
\centering
\includegraphics[width=160mm,frame]{graphs/tsapp_screenshot.jpg}
\vspace{-10pt}
\caption{The web interface of our system \label{fig:interface}}
\vspace{-20pt}
\end{figure*}
%to add: summary of contributions (phrase as 'implemented' not as 'in this paper')
%overview of demo. what people will see in the demo. 


%\balance
{\small
\bibliographystyle{abbrv}
\bibliography{TableSummarizationDemo}  
}

% Example of an appendix; typically would start on a new page
%pagebreak

\techreporttext{\input{appendix}}

\end{document}
