%!TEX root = TableSummarizationDemo.tex

\section{Summary of Contributions}\label{sec:summary}

Our system consists of two main components. The first component determines what rules to display to a user based on the user's latest interaction, and the values of parameters such as number of rules to display ($k$), weighting function $W$ to use, and so on. In order to do this, it has to make a pass through the table data several times. This can be expensive for big tables, so we dynamically maintain multiple samples of different parts of the table in memory instead. The second component is responsible for maintaining samples in memory and updating them when required. We now describe these two components in further detail below.

The problem of choosing the optimal rule list of a given size, is NP-Hard. However, we find an approximately optimal solution as follows: We first notice that given a set of rules, a rule-list consisting of those rules has the highest score if the rules are sorted in decreasing order by weight. So we can define the score of a rule {\em set} to be the score of the rule-list obtained by ordering rules of the set in decreasing order by weight. Thus our problem reduces to that of finding the highest scoring rule set. As long as the weight function is monotonic, the score of a rule-set can be shown to be submodular. Then we use the fact that a submodular function can be optimized using a greedy algorithm to choose rules one at a time in a greedy manner until we have $k$ rules. We call the above algorithm BRS (which stands for {\bf B}est {\bf R}ule {\bf S}et). Additional details on our approximation algorithm can be found in our technical report~\cite{tr}. 

The second component of our system is called the `SampleHandler'. TO begin with, it takes two user-specified input parameters: the memory capacity $M$, and a parameter called $minSS$. $minSS$ determines the sample size required to run the BRS algorithm. Higher values of minSS increase processing time (since the algorithm has to process a larger amount of data) but also increase the accuracy of the resulting displayed rule-list and rule counts. 


% algorithm for choosing rules
% samplehandler for managing memory
% our datasets
