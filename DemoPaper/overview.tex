%!TEX root = TableSummarizationDemo.tex
\vspace{-10pt}
\section{Demonstration Overview} \label{sec:demo} 
In our demonstration, we will show our prototype implementation of a system equipped with smart drill-down. We first describe the prototype implementation, the datasets, and then describe the demonstration scenarios.

\stitle{Prototype Details.} Our system is built as a web-application using NodeJS~\cite{nodejs}, with ExpressJS for the back-end, and AngularJS~\cite{angular} for the front-end. The Rule Finder and Sample Handler components (displayed in Figure~\ref{fig:system}) are coded in Java, and converted into an executable .jar that gets called by the web server backend. 

\stitle{Dataset Description.} In our demonstration, we will use two datasets. The first is an example of a dataset that attendees are not likely to be very familiar with, and the second is an example of a dataset that attendees are likely to be familiar with:

The first dataset, denoted `Marketing', contains demographic information about potential customers~\cite{dataset1}. A total of $9409$ questionnaires containing $502$ questions were filled out by shopping mall customers in the San Fransisco Bay Area; the dataset is a summary of their responses. Each tuple in the dataset describes a single person, with attributes such as gender, marital status, age, and so on. 

The second dataset is a US 1990 Census dataset from the UCI Machine Learning Repository~\cite{uciml}. It has 2.5 million tuples, with each tuple corresponding to a person. %In both the datasets, numerical columns such as age have been bucketized beforehand. 


\stitle{Demonstration Scenarios.} The goal of demonstration is to (a) illustrate
the utility of smart drill-down, along with the various possible interactions, 
in comparison with traditional drill-down; 
(b) demonstrate the effect on performance and utility on varying parameters of the Rule Finder, as controlled by the user interface; here the attendees will select different settings and examine the response; and
(c) demonstrate the effect on performance and accuracy on varying sampling parameters of the Sample Handler, in-built in the system.   

During the demo, we will set up instances of the system with the two previously described datasets pre-loaded. We will also have the web user interface open in a browser window. Then for each demonstration, we will go though three scenarios one after the other:
\squishlist
\item {\bf Scenario 1: Comparison to drill-down}: To begin with, we will have some canned exploration scenarios in order to familiarize attendees with the system interface and its adjustable parameters. Through the scenarios, we will demonstrate how smart drill-down lets one discover interesting information about a table efficiently, and how the table exploration can be tailored to fit a user's interests. These scenarios will highlight the advantage of smart drill-down compared with traditional drill-down. 
\item {\bf Scenario 2: Rule Finder Parameters}: We will allow the attendees to vary parameters shown in the user interface and observe their effects on the rules displayed, as well as response time and accuracy. Increasing the `Number of Rules' parameter will result in a longer rule list being displayed, but will cause an increase in response time. Reducing the `Max Weight' parameter, which allows the system to ignore rules having weight higher than the Max Weight, will speed up the response time of the system, but reducing it too much will result in a suboptimal rule-list being displayed. The third parameter, the weighting function, determines which rules the user finds `interesting'. Using a different weighting function, such as the `Bits' (which gives higher weight to rules containing non-$\star$ values in columns that have a large number of distinct values) will prioritize columns such as `Education' over columns such as `Gender' (since the latter has only two distinct values). The user will also be able to ignore certain columns, or force certain columns to be instantiated in the displayed rules. 
\item {\bf Scenario 3: Sample Handler Parameters}: We will allow attendees to try out multiple instances of the system initialized with different values of the $minSS$  parameter (recall that $minSS$ is the minimum sample size used by the system when determining which rules to display). This will allow attendees to observe how decreasing $minSS$ decreases running time but also potentially reduces accuracy of the displayed rules and their counts.
\squishend
\noindent With our demonstration, we hope to convince attendees that smart drill-down offers
a valuable alternative to traditional drill-down in quickly ``zooming into'' the interesting portions of a dataset.