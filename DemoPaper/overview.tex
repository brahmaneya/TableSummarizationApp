%!TEX root = TableSummarizationDemo.tex

\section{Demonstration Overview} \label{sec:demo} 
In our demonstration, we will show our prototype implementation of a system equipped with smart drill-down. We first describe the prototype implementation, the datasets, and then describe the demonstration scenarios.

\stitle{Prototype Details.} \agp{Manas, could you fill out some details here.} Our system is built as a web-application on XXX~\cite{XXX}, using XXX as the back-end, and XXX as the front-end. The components (as displayed in Figure~\ref{fig:system}) are coded in XXX. \agp{MORE DETAILS.}

\stitle{Dataset Description.} In our demonstration, we will use to datasets,
a marketing survey dataset \agp{Manas, provide some details of what this is.}, and a
US Census dataset. 
The former would be an example of a dataset that users are not very familiar with,
and the latter would be an example of a dataset that users are familiar with.


\stitle{Demonstration Scenarios.} We will have three demonstration scenarios:
\agp{More details on this the better. Try to add as much as you can..}
\squishlist
\item {\bf Scenario 1: Comparison to drill-down}: We will have some canned exploration scenarios \agp{why canned? Why not let attendees browse what they want?}  where we demonstrate how smart drill down lets one discover interesting information about a table efficiently. These scenarios will highlight the advantage of smart drill down compared with traditional drill down. 
\item {\bf Scenario 2: Rule Finder Parameters}: We will allow the user to vary parameters shown in the user interface, and observe their effects on the rules displayed, as well as the response time and accuracy. \agp{describe what these parameters are.} For instance, using smaller values for max weight will speed up the response, but a very small value will result in the user seeing an non-optimal table summary. Using a different weighting function, such as `Bits' \agp{have you described what this is?} will prioritize columns with a larger number of distinct values.
\item {\bf Scenario 3: Sample Handler Parameters}: \agp{describe what MinSS is.} We will allow users to try out instances of the system with different values of $minSS$ and let them observe the resulting effects on response time and accuracy.
\squishend
\agp{The screenshots are not very clear. Could you generate PDFs instead of jpgs? That may help. }